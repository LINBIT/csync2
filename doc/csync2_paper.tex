\documentclass[a4paper,twocolumn]{article}
\usepackage{nopageno}

\def\csync2{{\sc Csync$^{2}$}}

\begin{document}

\title{Cluster synchronization with \csync2}
\author{Original Author: Clifford Wolf for LINBIT\\
Developed and maintained by LINBIT, http://www.linbit.com/}
\maketitle

\section{Introduction}

\csync2 [1] is a tool for asynchronous file synchronization in clusters.
Asynchronous file synchronization is good for files which are seldom modified -
such as configuration files or application images - but it is not adequate for
some other types of data.

For instance a database with continuous write accesses should be synced
synchronously in order to ensure the data integrity. But that does not
automatically mean that synchronous synchronization is better; it simply is
different and there are many cases where asynchronous synchronization is
favored over synchronous synchronization. Some pros of asynchronous
synchronization are:

{\bf 1.}
Most asynchronous synchronization tools (including \csync2) are implemented as
single-shot commands which need to be executed each time in order to run one
synchronization cycle. Therefore it is possible to test changes on one host
before deploying them on the others (and also return to the old state if the
changes turn out to be bogus).

{\bf 2.}
The synchronization algorithms are much simpler and thus less error-prone.

{\bf 3.}
Asynchronous synchronization tools can be (and usually are) implemented as
normal user mode programs. Synchronous synchronization tools need to be
implemented as operating system extensions. Therefore asynchronous tools are
easier to deploy and more portable.

{\bf 4.}
It is much easier to build systems which allow setups with many hosts and
complex replication rules.

But most asynchronous synchronization tools are pretty primitive and do not
even cover a small portion of the issues found in real world environments.

I have developed \csync2 because I found none of the existing tools for
asynchronous synchronization satisfying. The development of \csync2 has
been sponsored by LINBIT Information Technologies [2], the company which also
sponsors the synchronous block device synchronization toolchain DRBD [3].

\hspace{0.2cm}

Note: I will simply use the term {\it synchronization} instead of the
semi-oxymoron {\it asynchronous synchronization} in the rest of this paper.

\subsection{\csync2 features}

Most synchronization tools are very simple wrappers for remote-copy tools such
as {\tt rsync} or {\tt scp}. These solutions work well in most cases but
still leave a big gap for more sophisticated tools such as \csync2. The most
important features of \csync2 are described in the following sections.

\subsubsection{Conflict detection}

\label{confl_detect}

Most of the trivial synchronization tools just copy the newer file over the
older one. This can be a very dangerous behavior if the same file has been
changed on more than one host. \csync2 detects such a situation as a conflict
and will not synchronize the file.  Those conflicts then need to be resolved
manually by the cluster administrator.

It is not considered as a conflict by \csync2 when the same change has been
performed on two hosts (e.g. because it has already been synchronized with
another tool).

It is also possible to let \csync2 resolve conflicts automatically for some or
all files using one of the pre-defined auto-resolve methods. The available
methods are: {\tt none} (the default behavior), {\tt first} (the host on which
\csync2 is executed first wins), {\tt younger} and {\tt older} (the younger or
older file wins), {\tt bigger} and {\tt smaller} (the bigger or smaller file
wins), {\tt left} and {\tt right} (the host on the left side or the right side
in the host list wins).

The {\tt younger}, {\tt older}, {\tt bigger} and {\tt smaller} methods let the
remote side win the conflict if the file has been removed on the local side.

\subsubsection{Replicating file removals}

Many synchronization tools can not synchronize file removals because they can
not distinguish between the file being removed on one host and being created on
the other one. So instead of removing the file on the second host they recreate
it on the first one.

\csync2 detects file removals as such and can synchronize them correctly.

\subsubsection{Complex setups}

Many synchronization tools are strictly designed for two-host-setups. This is
an inadequate restriction and so \csync2 can handle any number of hosts.

\csync2 can even handle complex setups where e.g. all hosts in a cluster share
the {\tt /etc/hosts} file, but one {\tt /etc/passwd} file is only shared among
the members of a small sub-group of hosts and another {\tt /etc/passwd} file is
shared among the other hosts in the cluster.

\subsubsection{Reacting to updates}

In many cases it is not enough to simply synchronize a file between cluster
nodes. It also is important to tell the applications using the synchronized
file that the underlying file has been changed, e.g. by restarting the
application.

\csync2 can be configured to execute arbitrary commands when files matching an
arbitrary set of shell patterns are synchronized.

\section{The \csync2 algorithm}

Many other synchronization tools compare the hosts, try to figure out which
host is the most up-to-date one and then synchronize the state from this host
to all other hosts. This algorithm can not detect conflicts, can not
distinguish between file removals and file creations and therfore it is not
used in \csync2.

\csync2 creates a little database with filesystem metadata on each host. This
database ({\tt /var/lib/csync2/{\it hostname}.db}) contains a list of the local
files under the control of \csync2. The database also contains information such
as the file modification timestamps and file sizes.

This database is used by \csync2 to detect changes by comparison with the local
filesystem. The synchronization itself is then performed using the \csync2
protocol (TCP port 30865).

Note that this approach implies that \csync2 can only push changes from the
machine on which the changes has been performed to the other machines in the
cluster. Running \csync2 on any other machine in the cluster can not detect and
so can not synchronize the changes.

Librsync [4] is used for bandwidth-saving file synchronization and SSL is used for
encrypting the network traffic. The sqlite library [5] (version 2) is used for
managing the \csync2 database files. Authentication is performed using
auto-generated pre-shared-keys in combination with the peer IP address and
the peer SSL certificate.

\section{Setting up \csync2}

\subsection{Building \csync2 from source}

Simply download the latest \csync2 source tar.gz from {\bf \tt http://oss.linbit.com/csync2/},
extract it and run the usual {\tt ./configure} - {\tt make} - {\tt make install} trio.

\csync2 has a few prerequisites in addition to a C compiler, the standard
system libraries and headers and the usual gnu toolchain ({\tt make}, etc):

{\bf 1.} You need librsync, libsqlite (version 2) and gnutls installed
(including development headers).

{\bf 2.} Bison and flex are needed to build the configuration file parser.

\subsection{\csync2 in Linux distributions}

As of November 2011, all major linux distributions have some \csync2 1.34 package
available, but to upgrade to an up-to-date \csync2 version, you probably need
to wait a bit or build the package from source.

The \csync2 source package contains an RPM {\tt .spec} file as well as a {\tt
debian/} directory. So it is possible to use {\tt rpmbuild} or {\tt debuild} to
build \csync2.

\subsection{Post installation}

Next you need to create an SSL certificate for the local \csync2 server.
Simply running {\tt make cert} in the \csync2 source directory will create and
install a self-signed SSL certificate for you.  Alternatively, if you have no
source around, run the following commands:

\begin{verbatim}
openssl genrsa \
    -out /etc/csync2_ssl_key.pem 1024
openssl req -new \
    -key /etc/csync2_ssl_key.pem \
    -out /etc/csync2_ssl_cert.csr
openssl x509 -req -days 600 \
    -in /etc/csync2_ssl_cert.csr \
    -signkey /etc/csync2_ssl_key.pem \
    -out /etc/csync2_ssl_cert.pem
\end{verbatim}

You have to do that on each host you're running csync2 on. When servers are
talking with each other for the first time, they add each other to the database.

The \csync2 TCP port 30865 needs to be added to the {\tt /etc/services} file and
inetd needs to be told about \csync2 by adding

\begin{verbatim}
csync2 stream tcp nowait root \
        /usr/local/sbin/csync2 csync2 -i
\end{verbatim}

to {\tt /etc/inetd.conf}.

%%%%%%%%%%%%%%%%%%%%%%%%%%%%%%%%%%%%%%%%%%%%%%%%%%%%%%%%%%%%%%%%%%%%%%%%%%%%%%%

\begin{figure*}[t]
  \begin{center}
\begin{verbatim}
group mygroup                           # A synchronization group (see 3.4.1)
{
        host host1 host2 (host3);       # host list (see 3.4.2)
        host host4@host4-eth2;

        key /etc/csync2.key_mygroup;    # pre-shared-key (see 3.4.3)

        include /etc/apache;            # include/exclude patterns (see 3.4.4)
        include %homedir%/bob;
        exclude %homedir%/bob/temp;
        exclude *~ .*;

        action                          # an action section (see 3.4.5)
        {
                pattern /etc/apache/httpd.conf;
                pattern /etc/apache/sites-available/*;
                exec "/usr/sbin/apache2ctl graceful";
                logfile "/var/log/csync2_action.log";
                do-local;
                # do-local-only;
        }

        backup-directory /var/backups/csync2;
        backup-generations 3;           # backup old files (see 3.4.11)

        auto none;                      # auto resolving mode (see 3.4.6)
}

prefix homedir                          # a prefix declaration (see 3.4.7)
{
        on host[12]: /export/users;
        on *:        /home;
}
\end{verbatim}
  \end{center}
  \caption{Example \csync2 configuration file}
\end{figure*}

%%%%%%%%%%%%%%%%%%%%%%%%%%%%%%%%%%%%%%%%%%%%%%%%%%%%%%%%%%%%%%%%%%%%%%%%%%%%%%%

%%%%%%%%%%%%%%%%%%%%%%%%%%%%%%%%%%%%%%%%%%%%%%%%%%%%%%%%%%%%%%%%%%%%%%%%%%%%%%%

\begin{figure*}[t]
  \begin{center}
\begin{verbatim}
csync2 -cr /
if csync2 -M; then
        echo "!!"
        echo "!! There are unsynced changes! Type 'yes' if you still want to"
        echo "!! exit (or press crtl-c) and anything else if you want to start"
        echo "!! a new login shell instead."
        echo "!!"
        if read -p "Do you really want to logout? " in &&
           [ ".$in" != ".yes" ]; then
                exec bash --login
        fi
fi
\end{verbatim}
  \end{center}
  \caption{The {\tt csync2\_locheck.sh} script}
\end{figure*}

%%%%%%%%%%%%%%%%%%%%%%%%%%%%%%%%%%%%%%%%%%%%%%%%%%%%%%%%%%%%%%%%%%%%%%%%%%%%%%%

\subsection{Configuration File}

Figure 1 shows a simple \csync2 configuration file. The configuration filename
is {\tt /etc/csync2.cfg} when no {\tt -C {\it configname}} option has been
passed and {\tt /etc/csync2\_{\it configname}.cfg} with a {\tt -C {\it
configname}} option.

\subsubsection{Synchronization Groups}

In the example configuration file you will find the declaration of a
synchronization group called {\tt mygroup}. A \csync2 setup can have any number
of synchronization groups. Each group has its own list of member hosts and
include/exclude rules.

\csync2 automatically ignores all groups which do not contain the local
hostname in the host list. This way you can use one big \csync2 configuration
file for the entire cluster.

\subsubsection{Host Lists}

Host lists are specified using the {\tt host} keyword. You can eighter specify
the hosts in a whitespace seperated list or use an extra {\tt host} statement
for each host.

The hostnames used here must be the local hostnames of the cluster nodes. That
means you must use exactly the same string as printed out by the {\tt hostname}
command. Otherwise csync2 would be unable to associate the hostnames in the
configuration file with the cluster nodes.

The {\tt -N \it hostname} command line option can be used to set the local
hostname used by \csync2 to a different value than the one provided by the {\tt
hostname} command. This may be e.g. useful for environments where the local
hostnames are automatically set by a DHCP server and because of that change
often.

Sometimes it is desired that a host is receiving \csync2 connections on an IP
address which is not the IP address its DNS entry resolves to, e.g.~when a
crossover cable is used to directly connect the hosts or an extra
synchronization network should be used. In this cases the syntax {\tt{\it
hostname}@{\it interfacename}} has to be used for the {\tt host} records (see
{\tt host4} in the example config file).

Sometimes a host shall only receive updates from other hosts in the
synchronization group but shall not be allowed to send updates to the other
hosts. Such hosts (so-called {\it slave hosts}) must be specified in
brackets, such as {\tt host3} in the example config file.

\subsubsection{Pre-Shared-Keys}

Authentication is performed using the IP addresses and pre-shared-keys in
\csync2. Each synchronization group in the config file must have exactly one
{\tt key} record specifying the file containing the pre-shared-key for this
group. It is recommended to use a separate key for each synchronization group
and only place a key file on those hosts which actually are members in the
corresponding synchronization group.

The key files can be generated with {\tt csync2 -k {\it filename}}.

\subsubsection{Include/Exclude Patterns}

The {\tt include} and {\tt exclude} patterns are used to specify which files
should be synced in the synchronization group.

There are two kinds of patterns: pathname patterns which start with a slash
character (or a prefix such as the {\tt \%homedir\%} in the example; prefixes
are explained in a later section) and basename patterns which do not.

The last matching pattern for each of both categories is chosen. If
both categories match, the file will be synchronized.

The pathname patterns are matched against the beginning of the filename. So they
must either match the full absolute filename or must match a directory in the
path to the file. The file will not be synchronized if no matching {\tt include} or
{\tt exclude} pathname pattern is found (i.e. the default pathname pattern is
an exclude pattern).

The basename patterns are matched against the base filename without the path. So
they can e.g. be used to include or exclude files by their filename extensions.
The default basename pattern is an include pattern.

In our example config file that means that all files from {\tt /etc/apache} and
{\tt \%homedir\%/bob} are synced, except the dot files, files with a tilde
character at the end of the filename, and files from {\tt
\%homedir\%/bob/temp}.

\subsubsection{Actions}

Each synchronization group may have any number of {\tt action} sections. These
{\tt action} sections are used to specify shell commands which should be
executed after a file is synchronized that matches any of the specified
patterns.

The {\tt exec} statement is used to specify the command which should be
executed. Note that if multiple files matching the pattern are synced in one
run, this command will only be executed once. The special token {\tt \%\%} in
the command string is substituted with the list of files which triggered the
command execution.

The output of the command is appended to the specified logfile, or to
{\tt /dev/null} if the {\tt logfile} statement is omitted.

Usually the action is only triggered on the targed hosts, not on the host on
which the file modification has been detected in the first place. The {\tt
do-local} statement can be used to change this behavior and let \csync2 also
execute the command on the host from which the modification originated. You can 
use {\tt do-local-only} to execute the action only on this machine.

\subsubsection{Conflict Auto-resolving}

The {\tt auto} statement is used to specify the conflict auto-resolving
mechanism for this synchronization group. The default value is {\tt auto none}.

See section \ref{confl_detect} for a list of possible values for this setting.

\subsubsection{Prefix Declarations}

Prefixes (such as the {\tt \%homedir\%} prefix in the example configuration
file) can be used to synchronize directories which are named differently on
the cluster nodes. In the example configuration file the directory for the
user home directories is {\tt /export/users} on the hosts {\tt host1} and
{\tt host2} and {\tt /home} on the other hosts.

The prefix value must be an absolute path name and must not contain any
wildcard characters.

\subsubsection{The {\tt nossl} statement}

Usually all \csync2 network communication is encrypted using SSL. This can be
changed with the {\tt nossl} statement. This statement may only occur in the
root context (not in a {\tt group} or {\tt prefix} section) and has two
parameters. The first one is a shell pattern matching the source DNS name for
the TCP connection and the second one is a shell pattern matching the
destination DNS name.

So if e.g.~a secure synchronization network is used between some hosts and
all the interface DNS names end with {\tt -sync}, a simple

\begin{verbatim}
nossl *-sync *-sync;
\end{verbatim}

will disable the encryption overhead on the synchronization network. All other
traffic will stay SSL encrypted.

\subsubsection{The {\tt config} statement}

The {\tt config} statement is nothing more then an include statement and can be
used to include other config files. This can be used to modularize the
configuration file.

\subsubsection{The {\tt ignore} statement}

The {\tt ignore} statement can be used to tell \csync2 to not check and not sync
the file user-id, the file group-id and/or the file permissions. The statement
is only valid in the root context and accepts the parameters {\tt uid}, {\tt
gid} and {\tt mod} to turn off handling of user-ids, group-ids and file
permissions.

\subsubsection{The {\tt tempdir} statement}

Preferably don't use this setting.

The {\tt tempdir} statement specifies the directory to be used for temporary
files while receiving data through librsync.

\csync2 will try to create temporary files in {\tt tempdir} if specified,
in the same directory as the currently processed file,
in the directory given by the {\tt TMPDIR} environment variable,
the system default {\tt P\_tmpdir}, or {\tt /tmp}, in that order.

This implies that if you specify a {\tt tempdir} which is not on the
same file system as the processed files, it will be impossible to
rename the patched files in place, and \csync2 will fall back to truncate
and copy. Which can in various failure scenarios result in corrupted (partially
transfered, truncated) files on the destination.

\subsubsection{The {\tt lock-timeout} statement}

The {\tt lock-timeout} statement specifies the seconds to wait wor a database lock
before giving up. Default is 12 seconds. The amount will be slightly randomized
with a jitter of up to 6 seconds based on the respective process id.

%%%%%%%%%%%%%%%%%%%%%%%%%%%%%%%%%%%%%%%%%%%%%%%%%%%%%%%%%%%%%%%%%%%%%%%%%%%%%%%

\begin{figure*}[t]
  \begin{center}
\begin{verbatim}
CREATE TABLE file (
        filename, checktxt,
        UNIQUE ( filename ) ON CONFLICT REPLACE
);

CREATE TABLE dirty (
        filename, force, myname, peername,
        UNIQUE ( filename, peername ) ON CONFLICT IGNORE
);

CREATE TABLE hint (
        filename, recursive,
        UNIQUE ( filename, recursive ) ON CONFLICT IGNORE
);

CREATE TABLE action (
        filename, command, logfile,
        UNIQUE ( filename, command ) ON CONFLICT IGNORE
);

CREATE TABLE x509_cert (
        peername, certdata,
        UNIQUE ( peername ) ON CONFLICT IGNORE
);
\end{verbatim}
  \end{center}
  \caption{The \csync2 database schema}
\end{figure*}

%%%%%%%%%%%%%%%%%%%%%%%%%%%%%%%%%%%%%%%%%%%%%%%%%%%%%%%%%%%%%%%%%%%%%%%%%%%%%%%

\subsubsection{Backing up}

\csync2 can back up the files it modifies. This may be useful for scenarios
where one is afraid of accidentally syncing files in the wrong direction.

The {\tt backup-directory} statement is used to tell \csync2 in which directory
it should create the backup files and the {\tt backup-generations} statement is
used to tell \csync2 how many old versions of the files should be kept in the
backup directory.

The files in the backup directory are named like the file they back up, with
all slashes substituted by underscores and a generation counter appended. Note
that only the file content, not the metadata such as ownership and permissions
are backed up.

Per default \csync2 does not back up the files it modifies. The default
value for {\tt backup-generations} is {\tt 3}.

\subsection{Activating the Logout Check}

The \csync2 sources contain a little script called {\tt csync2\_locheck.sh} 
(Figure 2).

If you copy that script into your {\tt \textasciitilde/.bash\_logout} script
(or include it using the {\tt source} shell command), the shell will not let
you log out if there are any unsynced changes.

\section{Database Schema}

Figure 3 shows the \csync2 database schema. The database can be accessed using
the {\tt sqlite} command line shell. All string values are URL encoded in the
database.

The {\tt file} table contains a list of all local files under \csync2 control,
the {\tt checktxt} attribute contains a special string with information about
file type, size, modification time and more. It looks like this:

\begin{verbatim}
v1:mtime=1103471832:mode=33152:
uid=1001:gid=111:type=reg:size=301
\end{verbatim}

This {\tt checktxt} attribute is used to check if a file has been changed on
the local host.

If a local change has been detected, the entry in the {\tt file} table is
updated and entries in the {\tt dirty} table are created for all peer hosts
which should be updated. This way the information that a host should be updated
does not get lost, even if the host in question is unreachable right now. The
{\tt force} attribute is set to {\tt 0} by default and to {\tt 1} when the
cluster administrator marks one side as the right one in a synchronization
conflict.

The {\tt hint} table is usually not used. In large setups this table can be
filled by a daemon listening on the inotify API. It is possible to tell \csync2
to not check all files it is responsible for but only those which have entries
in the {\tt hint} table. However, the Linux syscall API is so fast that this
only makes sense for really huge setups.

The {\tt action} table is used for scheduling actions. Usually this table is
empty after \csync2 has been terminated. However, it is possible that \csync2
gets interrupted in the middle of the synchronization process. In this case
the records in the {\tt action} table are processed when \csync2 is executed
the next time.

The {\tt x509\_cert} table is used to cache the SSL cetrificates used by the
other hosts in the csync2 cluster (like the SSH {\tt known\_hosts} file).

\section{Running \csync2}

%%%%%%%%%%%%%%%%%%%%%%%%%%%%%%%%%%%%%%%%%%%%%%%%%%%%%%%%%%%%%%%%%%%%%%%%%%%%%%%

\begin{figure*}[t]
  \begin{center}
    \begin{tabular}{|p{0.5\linewidth}|p{0.5\linewidth}|}
      \hline
\begin{tiny}
\begin{verbatim}
csync2 2.0 - cluster synchronization tool, 2nd generation
Copyright (C) 2004 - 2015 LINBIT Information Technologies GmbH
        http://www.linbit.com
See also: http://git.linbit.com/csync2.git/?a=blob;f=AUTHORS

Version: 2.0

This program is free software under the terms of the GNU GPL.

Usage: ./csync2 [-v..] [-C config-name] \
                [-D database-dir] [-N hostname] [-p port] ..

With file parameters:
        -h [-r] file..          Add (recursive) hints for check to db
        -c [-r] file..          Check files and maybe add to dirty db
        -u [-d] [-r] file..     Updates files if listed in dirty db
        -o [-r] file..          Create list of files in compare-mode
        -f [-r] file..          Force files to win next conflict resolution
        -m file..               Mark files in database as dirty

Simple mode:
        -x [-d] [[-r] file..]   Run checks for all given files and update
                                remote hosts.

Without file parameters:
        -c      Check all hints in db and eventually mark files as dirty
        -u [-d] Update (transfer dirty files to peers and mark as clear)

        -H      List all pending hints from status db
        -L      List all file-entries from status db
        -M      List all dirty files from status db

        -S myname peername      List file-entries from status db for this
                                synchronization pair.

        -T                      Test if everything is in sync with all peers.

        -T filename             Test if this file is in sync with all peers.

        -T myname peername      Test if this synchronization pair is in sync.

        -T myname peer file     Test only this file in this sync pair.

        -TT     As -T, but print the unified diffs.

        -i      Run in inetd server mode.
        -ii     Run in stand-alone server mode.
        -iii    Run in stand-alone server mode (one connect only).

        -R      Remove files from database which do not match config entries.

Exit codes:
        The modes -H, -L, -M and -S return 2 if the requested db is empty.
        The mode -T returns 2 if both hosts are in sync.
        Otherwise, only exit codes 0 (no errors)
        and 1 (some unspecified errrors) are expected.
\end{verbatim}
\end{tiny}

&

\begin{tiny}
\begin{verbatim}
Modifiers:
        -r      Recursive operation over subdirectories
        -d      Dry-run on all remote update operations

        -B      Do not block everything into big SQL transactions. This
                slows down csync2 but allows multiple csync2 processes to
                access the database at the same time. Use e.g. when slow
                lines are used or huge files are transferred.

        -A      Open database in asynchronous mode. This will cause data
                corruption if the operating system crashes or the computer
                loses power.

        -I      Init-run. Use with care and read the documentation first!
                You usually do not need this option unless you are
                initializing groups with really large file lists.

        -X      Also add removals to dirty db when doing a -TI run.
        -U      Don't mark all other peers as dirty when doing a -TI run.

        -G Group1,Group2,Group3,...
                Only use these groups from config-file.

        -P peer1,peer1,...
                Only update these peers (still mark all as dirty).
                Only show files for these peers in -o (compare) mode.

        -F      Add new entries to dirty database with force flag set.

        -t      Print timestamps to debug output (e.g. for profiling).

        -s filename
                Print timestamps also to this file.

        -W fd   Write a list of directories in which relevant files can be
                found to the specified file descriptor (when doing a -c run).
                The directory names in this output are zero-terminated.

Database switches:
        -D database-dir or url
                default: /var/lib/csync2
                Absolute path: use sqlite database in that directory
            URLs:
                sqlite:///some/path[/database.db3]
                sqlite3:///some/path[/database.db3]
                sqlite2:///some/path[/database.db]
                mysql://[<user>:<password>@]<hostname>/[database]
                pgsql://[<user>:<password>@]<hostname>/[database]
        If database is not given, it defaults to csync2_<qualified hostname>.
        Note that for non-sqlite backends, the database name is "cleaned",
        characters outside of [0-9][a-z][A-Z] will be replaced with _.

Creating key file:
        csync2 -k filename

Environment variables:
        CSYNC2_SYSTEM_DIR
                Directory containing csync2.cfg and other csync2 system files.
                Defaults to /etc.

Csync2 will refuse to do anything if this file is found:
$CSYNC2_SYSTEM_DIR/csync2.lock
\end{verbatim}
\end{tiny}
      \tabularnewline
      \hline
    \end{tabular}
  \end{center}
  \caption{The \csync2 help message}
\end{figure*}

%%%%%%%%%%%%%%%%%%%%%%%%%%%%%%%%%%%%%%%%%%%%%%%%%%%%%%%%%%%%%%%%%%%%%%%%%%%%%%%

Simply calling {\tt csync2} without any additional arguments prints out a
help message (Figure 4). A more detailed description of the most
important usage scenarios is given in the next sections.

\subsection{Just synchronizing the files}

The command {\tt csync2 -x} (or {\tt csync2 -xv}) checks for local changes and
tries to synchronize them to the other hosts. The option {\tt -d} (dry-run) can
be used to do everything but the actual synchronization.

When you start \csync2 the first time it compares its empty database with the
filesystem and sees that all files just have been created. It then will try
to synchronize the files. If the file is not present on the remote hosts it
will simply be copied to the other host. There also is no problem if the file
is already present on the remote host and has the same content. But if the
file already exists on the remote host and has a different content, you
have your first conflict.

\subsection{Resolving a conflict}

When two or more hosts in a \csync2 synchronization group have detected changes
for the same file we run into a conflict: \csync2 can not know which version is
the right one (unless an auto-resolving method has been specified in the
configuration file). The cluster administrator needs to tell \csync2 which
version is the correct one. This can be done using {\tt \csync2 -f}, e.g.:

\begin{verbatim}
# csync2 -x
While syncing file /etc/hosts:
ERROR from peer apollo:
    File is also marked dirty here!
Finished with 1 errors.

# csync2 -f /etc/hosts
# csync2 -xv
Connecting to host apollo (PLAIN) ...
Updating /etc/hosts on apollo ...
Finished with 0 errors.
\end{verbatim}

\subsection{Checking without syncing}

It is also possible to just check the local filesystem without doing any
connections to remote hosts: {\tt csync2 -cr /} (the {\tt -r} modifier
tells \csync2 to do a recursive check).

{\tt csync2 -c} without any additional parameters checks all files listed
in the {\tt hints} table.

The command {\tt csync2 -M} can be used to print the list of files marked dirty
and therfore scheduled for synchronization.

\subsection{Comparing the hosts}

The {\tt csync2 -T} command can be used to compare the local database with the
database of the remote hosts. Note that this command compares the databases and
not the filesystems - so make sure that the databases are up-to-date on all
hosts before running {\tt csync2 -T} and run {\tt csync2 -cr /} if you are
unsure.

The output of {\tt csync2 -T} is a table with 4 columns:

{\bf 1.} The type of the found difference: {\tt X} means that the file exists
on both hosts but is different, {\tt L} that the file is only present on the
local host and {\tt R} that the file is only present on the remote host.

{\bf 2.} The local interface DNS name (usually just the local hostname).

{\bf 3.} The remote interface DNS name (usually just the remote hostname).

{\bf 4.} The filename.

The {\tt csync2 -TT {\it filename}} command can be used for displaying unified
diffs between a local file and the remote hosts.

\subsection{Bootstrapping large setups}

The {\tt -I} option is a nice tool for bootstrapping larger \csync2
installations on slower networks. In such scenarios one usually wants to
initially replicate the data using a more efficient way and then use \csync2 to
synchronize the changes on a regular basis.

The problem here is that when you start \csync2 the first time it detects a lot
of newly created files and wants to synchronize them, just to find out that
they are already in sync with the peers.

The {\tt -I} option modifies the behavior of {\tt -c} so it only updates the
{\tt file} table but does not create entries in the {\tt dirty} table. So you
can simply use {\tt csync2 -cIr /} to initially create the \csync2 database on
the cluster nodes when you know for sure that the hosts are already in sync.

The {\tt -I} option may also be used with {\tt -T} to add the detected
differences to the dirty table and so induce \csync2 to synchronize the local
status of the files in question to the remote host.

Usually {\tt -TI} does only schedule local files which do exist to the dirty
database. That means that it does not induce \csync2 to remove a file on a
remote host if it does not exist on the local host. That behavior can be
changed using the {\tt -X} option.

The files scheduled to be synced by {\tt -TI} are usually scheduled to be
synced to all peers, not just the one peer which has been used in the {\tt -TI}
run. This behavior can be changed using the {\tt -U} option.

\subsection{Cleaning up the database}

It can happen that old data is left over in the \csync2 database after a
configuration change (e.g. files and hosts which are not referred anymore
by the configuration file). Running {\tt csync2 -R} cleans up such old
entries in the \csync2 database.

\subsection{Multiple Configurations}

Sometimes a higher abstracion level than simply having different
synchronization groups is needed. For such cases it is possible to use multiple
configuration files (and databases) side by side.

The additional configurations must have a unique name. The configuration file
is then named {\tt /etc/csync2\_{\it myname}.cfg} and the database is named
{\tt /var/lib/csync2/{\it hostname}\_{\it myname}.db}. Accordingly \csync2 must
be called with the {\tt -C {\it myname}} option.

But there is no need for multiple \csync2 daemons. The \csync2 protocol allows
the client to tell the server which configuration should be used for the
current TCP connection.

\section{Performance}

In most cases \csync2 is used for syncing just some (up to a few hundred) system
configuration files. In these cases all \csync2 calls are processed in less than
one second, even on slow hardware. So a performance analysis is not interesting
for these cases but only for setups where a huge amount of files is synced,
e.g. when syncing entire application images with \csync2.

A well-founded performance analysis which would allow meaningful comparisons
with other synchronization tools would be beyond the scope of this paper.
So here are just some quick and dirty numbers from a production
2-node cluster (2.40GHz dual-Xeon, 7200 RPM ATA HD, 1 GB Ram). The machines
had an average load of 0.3 (web and mail) during my tests..

I have about 128.000 files (1.7 GB) of Linux kernel sources and object
files on an ext3 filesystem under \csync2 control on the machines.

Checking for changes ({\tt csync2 -cr /}) took 13.7 seconds wall clock time,
9.1 seconds in user mode and 4.1 seconds in kernel mode. The remaining 0.5
seconds were spent in other processes.

Recreating the local database without adding the files to dirty table ({\tt
csync2 -cIr} after removing the database file) took 28.5 seconds (18.6 sec
user mode and 2.6 sec kernel mode). 

Comparing the \csync2 databases of both hosts ({\tt csync2 -T}) took 3 seconds
wall clock time.

Running {\tt csync2 -u} after adding all 128.000 files took 10 minutes wall
clock time. That means that \csync2 tried to sync all 128.000 files and then
recognized that the remote side had already the most up-to-date version of
the file after comparing the checksums.

All numbers are the average values of 10 iterations.

\section{Security Notes}

As statet earlier, authentication is performed using the peer IP address and a
pre-shared-key. The traffic is SSL encrypted and the SSL certificate of the
peer is checked when there has been already an SSL connection to that peer in
the past (i.e.~the peer certificate is already cached in the database).

All DNS names used in the \csync2 configuration file (the {\tt host} records)
should be resolvable via the {\tt /etc/hosts} file to guard against DNS
spoofing attacks.

Depending on the list of files being managed by \csync2, an intruder on one of
the cluster nodes can also modify the files under \csync2 control on the other
cluster nodes and so might also gain access on them. However, an intruder can
not modify any other files on the other hosts because \csync2 checks on the
receiving side if all updates are OK according to the configuration file.

For sure, an intruder would be able to work around this security checks when
\csync2 is also used to sync the \csync2 configuration files.

\csync2 only syncs the standard UNIX permissions (uid, gid and file mode).
ACLs, Linux ext2fs/ext3fs attributes and other extended filesystem permissions
are neither synced nor flushed (e.g. if they are set automatically when
the file is created).

On {\tt cygwin}, due to unresolved permission inheritance problems,
no {\tt rename()} is attempted, but existing files are always truncated
and then copied into from the temporary files.
Suggestions for how to resolve that are most welcome.

\section{Alternatives}

\csync2 is not the only file synchronization tool. Some of the other
free software file synchronization tools are:

\subsection{Rsync}

Rsync [6] is a tool for fast incremental file transfers, but is not a
synchronization tool in the context of this paper. Actually \csync2 is
using the rsync algorithm for file transfers. A variety of synchronization
tools have been written on top of rsync. Most of them are tiny shell scripts.

\subsection{Unison}

Unison [7] is using an algorithm similar to the one used by \csync2, but is
limited to two-host setups. Its focus is on interactive syncs (there even are
graphical user interfaces) and it is targeting on syncing home directories
between a laptop and a workstation. Unison is pretty intuitive to use, among
other things because of its limitations.

\subsection{Version Control Systems}

Version control systems such as Subversion [8] can also be used to synchronize
configuration files or application images. The advantage of version control
systems is that they can do three way merges and preserve the entire history
of a repository. The disadvantage is that they are much slower and require more
disk space than plain synchronization tools.

\section{References}

{[1]} \csync2 \\
http://oss.linbit.com/csync2/ \\
http://git.linbit.com/csync2.git/ \\
git://git.linbit.com/csync2/
\medskip \\
{[2]} LINBIT Information Technologies \\
http://www.linbit.com/
\medskip \\
{[3]} DRBD \\
http://www.drbd.org/
\medskip \\
{[4]} Librsync \\
http://librsync.sourceforge.net/
\medskip \\
{[5]} SQLite \\
http://www.sqlite.org/
\medskip \\
{[6]} Rsync \\
http://samba.anu.edu.au/rsync/
\medskip \\
{[7]} Unison \\
http://www.cis.upenn.edu/\textasciitilde{}bcpierce/unison/
\medskip \\
{[8]} Subversion \\
http://subversion.tigris.org/

\end{document}
